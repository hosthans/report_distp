\documentclass{article}


% if you need to pass options to natbib, use, e.g.:
%     \PassOptionsToPackage{numbers, compress}{natbib}
% before loading neurips_2023


% ready for submission
\usepackage[preprint]{neurips_2023}


% to compile a preprint version, e.g., for submission to arXiv, add add the
% [preprint] option:
%     \usepackage[preprint]{neurips_2023}


% to compile a camera-ready version, add the [final] option, e.g.:
%     \usepackage[final]{neurips_2023}


% to avoid loading the natbib package, add option nonatbib:
%    \usepackage[nonatbib]{neurips_2023}


\usepackage[utf8]{inputenc} % allow utf-8 input
\usepackage[T1]{fontenc}    % use 8-bit T1 fonts
\usepackage{hyperref}       % hyperlinks
\usepackage{url}            % simple URL typesetting
\usepackage{booktabs}       % professional-quality tables
\usepackage{amsfonts}       % blackboard math symbols
\usepackage{nicefrac}       % compact symbols for 1/2, etc.
\usepackage{microtype}      % microtypography
\usepackage{xcolor}         % colors
\usepackage{graphicx}



\title{Documentation Project DISTP}


% The \author macro works with any number of authors. There are two commands
% used to separate the names and addresses of multiple authors: \And and \AND.
%
% Using \And between authors leaves it to LaTeX to determine where to break the
% lines. Using \AND forces a line break at that point. So, if LaTeX puts 3 of 4
% authors names on the first line, and the last on the second line, try using
% \AND instead of \And before the third author name.


\author{%
  Group 3 \\
  Otto-Friedrich University Bamberg\\
  \texttt{\dots} \\ 
}


\begin{document}


\maketitle


\begin{abstract}
Diese Dokumentation und Designspezifikation beschreibt die Entwicklung einer medizinischen Anwendung zur kontinuierlichen Überwachung von Vorhofflimmern. Ziel ist es, Patient:innen eine zuverlässige und nutzerfreundliche Lösung bereitzustellen, die frühzeitig Anomalien erkennt und eine effiziente Kommunikation zwischen allen Beteiligten ermöglicht. Die Anwendung integriert Sensordaten, intelligente Algorithmen zur Analyse von Herzrhythmen sowie eine übersichtliche Benutzeroberfläche zur Darstellung relevanter Informationen. Durch den modularen Aufbau ist eine Erweiterung um zusätzliche Diagnose- und Monitoring-Funktionen möglich.
\end{abstract}

\section{Einleitung \& Zielsetzung}
Herz-Kreislauf-Erkrankungen wie Vorhofflimmern gehören zu den häufigsten Ursachen für Schlaganfälle und andere schwerwiegende Komplikationen. Regelmäßige Screenings können Leben retten – doch in der Praxis werden ihre Vorteile oft durch überfordernde Datenmengen, Fehlinterpretationen und die Verunsicherung der Betroffenen gemindert.

Unser Projekt stellt sich dieser Herausforderung mit einer klaren Vision: Eine App, die durch kontinuierliche und smarte Gesundheitsmessungen Sicherheit vermittelt – ohne zu verunsichern – und Ärzten relevante, kontextualisierte Informationen an die Hand gibt, um fundierte Entscheidungen zu erleichtern. Ein ausgemachtes Designziel war es, den Arbeitsaufwand für das medizinische Personal nicht zu erhöhen, da es bereits an der Belastungsgrenze arbeitet. Die App soll sowohl Patienten als auch dem medizinischen Fachpersonal einen spürbaren Mehrwert bieten und dabei ihre individuellen Bedürfnisse berücksichtigen.

\textbf{Für Patienten bedeutet das:}
\begin{itemize}
	\item Einfach zugängliche Messungen von Puls, Blutdruck und EKG über die Smartwatch.
	\item KI-gestützte Unterstützung durch einen Chatbot.
	\item Individuelle Medikamentenpläne und ein optionales Symptomtagebuch.
\end{itemize}

\textbf{Für Ärzte bietet die App:}
\begin{itemize}
	\item Klar aufbereitete, KI-unterstützte Messdaten in einer Reihe von möglichen Formaten.
	\item Zusatzinformationen zu Symptomen und Kontexten aus Patientensicht.
	\item Unterstützung bei der Aufklärung und Beruhigung der Patienten.
\end{itemize}

Unsere Gestaltung folgt den Prinzipien der Offenbacher Produktsprache: Große, selbsterklärende Bedienelemente, beruhigende Farbcodes (Blau als Hauptthema), sanfte Übergänge und eine klare, vertrauenswürdige Typografie. Sicherheit, Verständlichkeit und emotionale Entlastung sind die Kernelemente unseres Designs.

%\section{Projektkontext}
\subsection{Unser Team}
Unser Projektteams setzt sich aus einer interdisziplinären Gruppe von Studenten den Universität Bamberg zusammen.

\begin{itemize}
	\item \textbf{Anna Babicheva:} Wirtschaftsinformatik.
	\item \textbf{Hannes Weber:} Angewandte Informatik.
	\item \textbf{Benedikt Freiburg:} Computational Social Science.
	\item \textbf{Peter Geiger:} Computational Social Science.
\end{itemize}

Aufgrund der unterschiedlichen Studienrichtungen bringt jedes Projektmitglied individuelle Perspektiven und Skillsets mit. Da Anna bereits Erfahrungen mit Wireframes aus einem anderen Seminar hatte, übernahm sie die Erstellung der App-Wireframes. Sie war außerdem maßgeblich an der konzeptionellen Ausarbeitung der Personas und des Designs beteiligt und wirkte ebenso an dessen praktischen Umsetzung mit. Hannes war für die grundlegenden Designaufgaben verantwortlich und legte damit das Fundament für die spätere Verfeinerung durch Benedikt und Peter. Als technisch versiertestes Mitglied übernahm er zudem die Entwicklung technischer Lösungen für diverse Designprobleme. Benedikt und Peter kümmerten sich um die Detailarbeit, erstellten Screens und setzten Feedback um. Für Qualitätssicherung und Barrierefreiheit waren alle Mitglieder gemeinsam zuständig. Eine übergeordnete Projektleitung gab es nicht – wir arbeiteten in einer flachen, hierarchiefreien Struktur. Das verkürzte Kommunikationswege und erleichterte den Feedbackprozess.

\subsection{Projektumfang}
Unsere Kernfeatures belaufen sich auf:
\begin{itemize}
	\item Intuitiver Anmeldescreen.
	\item Blutdruck-, Puls- und EKG-Messungen sowie deren Interpretation.
	\item Kontaktfunktion für Angehörige / Notrufoption.
	\item Individuell einstellbare Medikamentenerinnerungen.
	\item Datenexport für Ärzte und Angehörige mit markanten Messwerten, Kontextkommentaren und Verlaufsdarstellung.
	\item Medi-KI-Chat, um schnelle Informationen mittels eines Chatbots zu erhalten und bei einfachen Fragen Ärztinnen und Ärzte zu entlasten.
	\item Möglichkeit für Angehörige, die Werte von Verwandten einzusehen.
	\item Korrespondierende UI für die App auf der Smartwatch.
\end{itemize}

Die sich aus diesen Kernfeatures ableitenden Funktionen sind:
\begin{itemize}
	\item Datenschutzeinstellungen zur Einhaltung der GDPR.
	\item Verschiedene Abstufungen für Mitteilungen und Benachrichtigungen.
	\item Automatische Messungen zum Erkennen von Mustern im Krankheitsbild.
	\item Erstellung eines Symptomtagebuchs.
\end{itemize}

\subsection{Rahmenbedingungen}
\begin{itemize}
	\item Offenbacher Produktsprache: Große, selbsterklärende Buttons, klare visuelle Rückmeldungen, intuitive Navigationslogik, beruhigender Sprachstil.
	\item Sicherheit: Eindeutige Visualisierung von Schutzmechanismen (z.B. Schloss-Icon).
	\item Barrierefreiheit: Klare Kontraste, ausreichend große Bedienelemente, einfache Sprache. Möglichkeit, zusätzliche bzw. weiterführende Informationen anzuzeigen, falls dies gewünscht ist.
	\item Styleguide: Farbschema dominiert von Blau (Sicherheit), ergänzt durch Weiß, Hellgrau und sanfte Blautöne. Dieses Farbschema wird nur selten durchbrochen, um z.B. Warnungen wie „Auffälligkeit erkannt“ (\texttt{FF8000}, Orange) deutlich darzustellen.
\end{itemize}

%\section{Designprinzipien}

\newpage

%\section{UI-Elemente}
\subsection{Buttons}
\begin{itemize}
	\item \textbf{Primär:} z.\,B.\ „Anmelden“, „Speichern“, „Weiter“, „Abbrechen“, „Fertig“.
	\item \textbf{Sekundär:} z.\,B.\ „Zurück“, „Erneut erinnern“, „Kommentar speichern“.
	\item \textbf{CTA (Call-to-Action):} z.\,B.\ „Notfallkontakt anrufen“, „Medikamentenplan öffnen“.
\end{itemize}

Primäre Buttons sind Blau ausgefüllt (\texttt{45B3CB}, Blau). Sekundäre Buttons sind nicht ausgefüllt, besitzen jedoch einen blauen Rand (\texttt{45B3CB}). Dies signalisiert dem Menschen welcher Button für den weiteren Prozess am wichtigsten ist. Sekundäre Buttons zeigen an, dass hier weitere Funktion und zusätzliche Informationen verfügbar sind.

\subsection{Input-Felder}
\begin{itemize}
	\item Textfelder für Telefonnummer, Code, Geburtsdatum, Symptombeschreibung.
	\item Datepicker für Zeit und Datum.
	\item Kommentartextfelder (max. 1000~Zeichen).
\end{itemize}

\subsection{Dropdowns \& Listen}
\begin{itemize}
	\item Dropdowns für Datumsauswahl, Messrhythmus (täglich, wöchentlich).
	\item Listen für Kontakte, Erinnerungen, Messwerte, Medikation.
\end{itemize}

\subsection{Navigation}
\begin{itemize}
	\item Hauptmenü mit Modulen wie Erinnerungen, Kontakte, Blutdruck \& Puls, Datenexport, Symptomtagebuch.
	\item Onboarding-Seiten für Registrierung, Einführung, Konfiguration.
\end{itemize}

\subsection{States (Normal, Hover, Disabled, Active)}
\begin{itemize}
	\item \textbf{Normal:} Standardzustand bei Buttons, Eingabefeldern.
	\item \textbf{Disabled:} Buttons ausgegraut, wenn Eingaben fehlen oder Funktion nicht verfügbar ist.
	\item \textbf{Active:} Aktiver Menüpunkt oder laufende Messung (z.\,B.\ Fortschrittsanzeige „30\,\%“).
\end{itemize}

\newpage

%\input{Sections/User Flows & Interaktionen.tex}
%\section{Barrierefreiheit}

\newpage

%\section{Technische Umsetzung}

\newpage

%\section{Learnings}

\newpage


%%%%%%%%%%%%%%%%%%%%%%%%%%%%%%%%%%%%%%%%%%%%%%%%%%%%%%%%%%%%

\bibliographystyle{plainnat}
\bibliography{bibliography}

\end{document}
