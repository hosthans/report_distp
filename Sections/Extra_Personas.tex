\section{Personas}
Die Personas wurden in dieser Form aufgebaut, um die verschiedenen Anforderungen, Verhaltensweisen und Erwartungen potenzieller Nutzer:innen unserer App systematisch zu umreißen und verständlich darzulegen. Jede Persona bündelt nicht nur demografische Daten, sondern vor allem Motivationen, Probleme („Pain Points“) und Erwartungen an die App. Dadurch entsteht ein Bild, wie unterschiedliche Zielgruppen mit der App arbeiten und welche Design- und Funktionsentscheidungen daraus erwachsen.

Der Aufbau folgt einem konsistenten Muster: Zunächst wurden Persönlichkeit, Lebensumstände und Interessen dargestellt. So ist ein authentisches, greifbares Bild der Person, das über reine Statistik hinausgeht entstanden. Elemente wie „Was ist in meiner Tasche?“ oder Lieblingsapps helfen, die Personas greifbar zu machen und typische Nutzungssituationen zu demonstrieren. Ebenso geben Abschnitte wie „Freunde \& Familie“ oder „Lebensstil“ Kontext, warum bestimmte Funktionen wichtig sind. So etwa Reminder, um dem Personas in ihrem Alltag zu helfen, oder einfache Bedienbarkeit, wenn wenig Technikaffinität vorhanden ist.

Ein Kernbestandteil ist der Umgang mit der Vorerkrankung. Hier wird deutlich, welche medizinischen Anforderungen gegeben sind und wie sich diese das alltägliche Leben beeinflussen. Daraus lassen sich die spezifischen Aufgaben (z. B. Medikamenteneinnahme, Puls- und Blutdruckkontrolle) sowie die Hauptbedenken der Personas (z.B. Stress, Angst vor Überforderung, falsche Einschätzung durch andere) ableiten. Diese Punkte zeigen, an welchen Stellen die App unterstützen und Sicherheit bieten muss.

Die „Patient Journeys“ bauen darauf auf, indem sie die Lebensabschnitte von den ersten Symptomen bis zur Langzeitbetreuung beschreiben. Jede Phase beinhaltet Gefühle, Verhaltensweisen, Pain Points und Erwartungen. Erkennbar wird so, wie sich die Bedürfnisse im Krankheitsverlauf. Für das Applications-Design ist das von enormer Wichtigkeit, da dies erlaubt, Funktionen angepasst auszurichten: Simple Erklärungen am Anfang, detailreiche Analyse für die Langzeitbetreuung, sowie intuitive Erinnerungen und Benachrichtigungen über den gesamten Zeitraum hinweg.
Zusammenfassend dient dieser Aufbau also dazu, eine Brücke zwischen medizinischen Anforderungen und dem tatsächlichen Alltag der Nutzer:innen zu schlagen. Personas und Patient Journeys machten so die abstrakten Zielgruppen konkret für unser Team erlebbar, deckten Unterschiede (z. B. junge, digitalaffine Nutzerin vs. älterer, technikferner Nutzer) auf und halfen, die App so zu gestalten, dass sie Sicherheit und Vertrauen schafft, Überforderung vermeidet und so hoffentlich langfristig genutzt wird.

\begin{figure}[h!]
	\centering
	\begin{minipage}{0.12\linewidth}
		\centering
		\includegraphics[width=\linewidth]{images/Fred}
		\caption*{Fred Goller}
	\end{minipage}
	\hfill
	\begin{minipage}{0.12\linewidth}
		\centering
		\includegraphics[width=\linewidth]{images/Selina}
		\caption*{Selina Neri}
	\end{minipage}
	\hfill
	\begin{minipage}{0.12\linewidth}
		\centering
		\includegraphics[width=\linewidth]{images/Horst}
		\caption*{Horst Breitenbach}
	\end{minipage}
	\hfill
	\begin{minipage}{0.12\linewidth}
		\centering
		\includegraphics[width=\linewidth]{images/Heike}
		\caption*{Heike Schneider}
	\end{minipage}
	\label{fig:personen}
\end{figure}

\newpage