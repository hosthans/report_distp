\section{Journey Map}
\subsection{Ziel der Journey Map}
Die Journey Map dient als generisches Modell des Krankheits- und Betreuungspfades für Patient:innen mit Vorhofflimmern. 
Sie unterteilt die User Experience in vier Hauptphasen. Anschliessend durchläuft die im Projekt definierte Persona jede dieser Phasen 
und bringt unterschiedliche Gefühle, Bedürfnisse und Erwartungen mit. 
All dies trägt zu einem tieferen Verständnis der Prozesse bei.

\subsection{Aufbau}
Die Map wurde in vier Analyseebenen unterteilt:
\begin{itemize}
	\item \textbf{Gefühle und Gedanken} -- emotionale Reaktionen in jeder Phase (z.B. Angst, Vertrauen)
	\item \textbf{Verhalten \& Bedürfnisse} -- welche Handlungen Nutzer:innen typischerweise zeigen (z.B. Arztbesuche, Medikamenteneinnahme)
	\item \textbf{Pain Points} -- kritische Momente, in denen Nutzer:innen besonders gefährdet sind, die App abzulehnen
	\item \textbf{Erwartungen an UI \& App-Funktionalitäten} -- abgeleitete Designanforderungen für jede Phase (z.B. visuelle Sicherheitssignale)
\end{itemize}

\subsection{Phasen}
\begin{description}
	\item[Phase 1: Beginn der Symptome]
	In dieser Phase werden die ersten Anzeichen und Unsicherheiten beschrieben, die die Betroffenen empfinden. 
	Ziel ist es, ihnen Orientierung zu geben und Ängste durch klare und einfache Informationen abzubauen.
	
	\item[Phase 2: Kontakt mit dem Gesundheitssystem \& Diagnose]
	Diese Phase ist den Arztbesuchen, Untersuchungen und der Erstdiagnose gewidmet. 
	Sie zielt darauf ab, Vertrauen aufzubauen und die Patienten bei der Aneignung neuer Gewohnheiten und der Einnahme von Medikamenten zu unterstützen.
	
	\item[Phase 3: Beginn der Behandlung]
	Therapie und regelmässige Messungen werden in dieser Phase Teil des Alltags. 
	Die App soll durch Automatisierung und Vereinfachung der Prozesse dazu beitragen, eine Überlastung zu vermeiden.
	
	\item[Phase 4: Langzeitbetreuung]
	Diese Phase beschreibt eine langfristige Unterstützung, bei der Gesundheit unauffällig in den Alltag integriert wird.
\end{description}
\newpage