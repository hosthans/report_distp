\section{User Flows \& Interaktionen}
\subsection{Klickpfade für typische Anwendungsfälle}
Die App hat klare, leicht verständliche Klickpfade für zentrale Anwendungsfälle wie Registrierung, Messungen, Medikamentenerinnerungen und Datenexport. Nutzer werden Schritt für Schritt durch die Abläufe geführt, mit eindeutig, grossen und beschrifteten Buttons sowie konsistenten Bedienelementen. Unser Figma-Projekt zeigt beispielhaft mehrere Standard-Use-Cases:

\begin{enumerate}
	\item \textbf{Registrierung \& Login}
	\begin{itemize}
		\item Startscreen $\rightarrow$ Sprachauswahl $\rightarrow$ Registrierungsmethode wählen
		\item Telefonnummer/Arztcode eingeben $\rightarrow$ SMS-Code
		\item Profil konfigurieren $\rightarrow$ Weiterführende Informationen $\rightarrow$ Hauptmenü
	\end{itemize}
	
	\item \textbf{Medikamentenplan}
	\begin{itemize}
		\item Hauptmenü $\rightarrow$ Medikamentenplan $\rightarrow$ Erinnerung hinzufügen
		\item Zeit \& Medikament wählen $\rightarrow$ Speichern
	\end{itemize}
	
	\item \textbf{Messungen (Puls/Blutdruck/EKG)}
	\begin{itemize}
		\item Hauptmenü $\rightarrow$ Messung starten $\rightarrow$ Messfortschritt (z.\,B. 30\%)
		\item Ergebnis: „Normal“, „Auffälligkeit erkannt“ oder „Erhöht“
		\item Daten speichern $\rightarrow$ Rückblick/Verlauf
	\end{itemize}
	
	\item \textbf{Datenexport}
	\begin{itemize}
		\item Hauptmenü $\rightarrow$ Datenexport $\rightarrow$ Zeitraum wählen $\rightarrow$ Dateiformat auswählen
		\item Export bestätigen $\rightarrow$ PDF/CSV generieren $\rightarrow$ Download/Teilen
		\item \emph{Alternative:} Hauptmenü $\rightarrow$ Datenexport $\rightarrow$ One-Click Export
	\end{itemize}
	
	\item \textbf{Notfallkontakt hinzufügen}
	\begin{itemize}
		\item Hauptmenü $\rightarrow$ Kontakte $\rightarrow$ „Notfallkontakt hinzufügen“
		\item Telefonnummer eingeben $\rightarrow$ Einladung versenden
		\item Kontakt sichtbar in Liste
	\end{itemize}
	
	\item \textbf{Medi-KI-Chat}
	\begin{itemize}
		\item Hauptmenü $\rightarrow$ Medi-KI-Chat
	\end{itemize}
	
	\item \textbf{Symptomtagebuch}
	\begin{itemize}
		\item Hauptmenü $\rightarrow$ Symptomtagebuch $\rightarrow$ Symptom eintragen $\rightarrow$ Eintrag speichern
	\end{itemize}
\end{enumerate}

\subsection{Feedback-Verhalten (Fehlermeldungen, Bestätigungen)}
Das Feedback-Verhalten nutzt klare Zeichen und Symbole. Meldungen sind präzise formuliert und geben Nutzern sofort verständliche Handlungsempfehlungen. Die UI zeigt mehrere Interaktionen mit Feedback:
\begin{itemize}
	\item \textbf{Bestätigungen:} „Messung erfolgreich durchgeführt“, „PDF wurde erstellt“
	\item \textbf{Warnungen:} Hinweise bei auffälligen Werten („Vorhofflimmern erkannt“), Medikamentenerinnerungen, Wetterhinweisen etc.
	\item Staffelungen der Warnungen in \textbf{vier Stufen}.
	\item \textbf{Handlungsempfehlungen:} „Bitte wiederholen Sie die Messung in Ruhe“ (häufig in Warnungen integriert), oft auch Hinweise auf Medi-KI-Chat.
\end{itemize}
