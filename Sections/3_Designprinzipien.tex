\section{Designprinzipien}
\subsection{Farbpalette mit Hex-Codes und Verwendungsrichtlinien}

\subsubsection{Primärfarbe}
\begin{itemize}
	\item \textbf{Blau} (\texttt{\#45B3CB}): Hauptbuttons, aktive Navigationselemente, CTAs, Schriften.  
	Blau wurde als Kernthema des Entwurfs gewählt, um den Fokus auf \textbf{Sicherheit} zu betonen. Zudem soll die Farbe beruhigend und schlicht wirken.
\end{itemize}

\subsubsection{Farbpalette}
\begin{tabular}{>{\columncolor{Primärblau}}m{2cm} m{6cm}}
	& \textbf{Primärblau} (\texttt{\#45B3CB}) – Hauptbuttons, aktive Navigation, CTAs. \\
\end{tabular}

\vspace{0.5em}

\begin{tabular}{>{\columncolor{Schwarz}}m{2cm} m{6cm}}
	& \textbf{Schwarz} (\texttt{\#000000}) – Textfarbe für Fliesstext. \\
\end{tabular}

\vspace{0.5em}

\begin{tabular}{>{\columncolor{Warnorange}}m{2cm} m{6cm}}
	& \textbf{Warnorange} (\texttt{\#FF8000}) – Warnhinweise, z.B. \textit{Auffälligkeit erkannt}. \\
\end{tabular}

\vspace{0.5em}

\begin{tabular}{>{\columncolor{Hellgrau}}m{2cm} m{6cm}}
	& \textbf{Hellgrau} (\texttt{\#F6F6F6}) – Hintergrundbereiche; Nicht-funktionale Buttons (Onboarding). \\
\end{tabular}

\vspace{0.5em}

\begin{tabular}{>{\columncolor{Tuerkisgruen}}m{2cm} m{6cm}}
	& \textbf{Tuerkisgruen} (\texttt{\#176272}) – Textfarbe in sekundären Buttons. \\
\end{tabular}

\vspace{0.5em}

Auch hier wurde bewusst eine beruhigende Farbpalette eingesetzt. Blau findet sich ebenfalls in den Texten wieder, während Weiss, Schwarz und Hellgrau für Klarheit und Lesbarkeit sorgen.

\subsubsection{Verwendung}
\begin{itemize}
	\item Primärfarben für Interaktionselemente und Überschriften.
	\item Neutralfarben für Hintergrund, Text und Layout.
\end{itemize}

\subsection{Typografie}

\subsubsection{Schriftarten}
\begin{itemize}
	\item \textbf{Überschriften:} Nunito Bold / Extra Bold
	\item \textbf{Fliesstext:} Nunito Regular / Bold
\end{itemize}

Gewählt haben wir die Schriftart Nunito, weil sie eine runde, freundliche und zugleich sehr gut lesbare Sans-Serif-Schrift darstellt. Sie ist modern, einladend und auf mobilen Displays ebenso gut erkennbar wie auf anderen, auch grösseren, Bildschirmen. Das schafft Vertrauen und Zugänglichkeit, besonders für ältere Nutzer.
Die gewählte hellblaue Farbpalette soll Sicherheit, Ruhe und Zuverlässigkeit symbolisieren. Das sind die zentrale Werte unserer Gesundheits-App. Blau lässt sich einfach mit Medizin und Technologie assoziieren. Das hellere Blau wirkt weniger streng als dunkle Töne und sorgt so für Offenheit und eine beruhigende Nutzererfahrung.

\begin{figure}[h!]
	\centering
	\includegraphics[width=0.7\linewidth]{images/Font}
	\caption{Beispieltext für Nunito.}
	\label{fig:font}
\end{figure}

\subsubsection{Grössen \& Anwendungsbereiche}
\begin{itemize}
	\item \textbf{H1} (\texttt{28--36\,px}): Screen-Titel, Hauptüberschriften
	\item \textbf{Body} (\texttt{20\,px}): Zwischenüberschriften
	\item \textbf{Small} (\texttt{15--18\,px}): Labels, Hilfstexte
\end{itemize}
Auch hier wurde stringent vorgegangen, um die App möglichst übersichtlich und einheitlich zu gestalten.

\newpage
