\section{Einleitung \& Zielsetzung}
Herz-Kreislauf-Erkrankungen wie Vorhofflimmern gehören zu den häufigsten Ursachen für Schlaganfälle und andere schwerwiegende Komplikationen. Regelmäßige Screenings können Leben retten – doch in der Praxis werden ihre Vorteile oft durch überfordernde Datenmengen, Fehlinterpretationen und die Verunsicherung der Betroffenen gemindert.

Unser Projekt stellt sich dieser Herausforderung mit einer klaren Vision: Eine App, die durch kontinuierliche und smarte Gesundheitsmessungen Sicherheit vermittelt – ohne zu verunsichern – und Ärzten relevante, kontextualisierte Informationen an die Hand gibt, um fundierte Entscheidungen zu erleichtern. Ein ausgemachtes Designziel war es, den Arbeitsaufwand für das medizinische Personal nicht zu erhöhen, da es bereits an der Belastungsgrenze arbeitet. Die App soll sowohl Patienten als auch dem medizinischen Fachpersonal einen spürbaren Mehrwert bieten und dabei ihre individuellen Bedürfnisse berücksichtigen.

\textbf{Für Patienten bedeutet das:}
\begin{itemize}
	\item Einfach zugängliche Messungen von Puls, Blutdruck und EKG über die Smartwatch.
	\item KI-gestützte Unterstützung durch einen Chatbot.
	\item Individuelle Medikamentenpläne und ein optionales Symptomtagebuch.
\end{itemize}

\textbf{Für Ärzte bietet die App:}
\begin{itemize}
	\item Klar aufbereitete, KI-unterstützte Messdaten in einer Reihe von möglichen Formaten.
	\item Zusatzinformationen zu Symptomen und Kontexten aus Patientensicht.
	\item Unterstützung bei der Aufklärung und Beruhigung der Patienten.
\end{itemize}

Unsere Gestaltung folgt den Prinzipien der Offenbacher Produktsprache: Große, selbsterklärende Bedienelemente, beruhigende Farbcodes (Blau als Hauptthema), sanfte Übergänge und eine klare, vertrauenswürdige Typografie. Sicherheit, Verständlichkeit und emotionale Entlastung sind die Kernelemente unseres Designs.
